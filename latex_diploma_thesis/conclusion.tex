\chapter*{Conclusion}
\addcontentsline{toc}{chapter}{Conclusion}

In this thesis, we presented two new algorithms that we implemented in an isometric reconciliation software which takes a rooted species tree with exact branch lengths and a rooted or unrooted gene tree with inexact branch lengths as an input. The output of our software is one or more reconciled gene trees with minimizing the number of duplications and gene losses.

We firstly analyse the current approach of solving the problem of reconciliation in general. We divided it into three main approaches: scoring, probabilistic and isometric gene tree reconciliation. We made an overview of existing software for the scoring and probabilistic reconciliation. We further focused on the isometric reconciliation which we subdivide into two sections. The first section is dealing with the isometric reconciliation with exact branch lengths introduced by Ma et al. \cite{ma} in 2008 and later corrected by Brejová et al. \cite{brejova}, who suggested extension for reconciliation of unrooted gene tree and species tree in running time $O(N^5 log N)$. The second section is isometric reconciliation with inexact branch lengths introduced by Chládek \cite{chladek_thesis} suggesting algorithms to infer the most parsimonious reconciliation for a rooted species tree and a semi-rooted or unrooted gene tree with the running time $O(N^4 log N)$.

We built on the algorithms from \cite{chladek_thesis} and created a new algorithm for rooting an unrooted gene tree. For each edge in the unrooted gene tree, it semi-roots the gene tree and subdivides the root edge by given step. The running time of the rooting algorithm is $O(N^2 + P)$. The second algorithm is for counting the evolutionary events in the gene tree and requires precomputed variables by the two-pass algorithm by \cite{chladek_thesis}. For each node in the gene tree, we count the duplications in the gene node and gene losses on the edge from the gene node to its parent. We split the algorithm into preprocessing and the main algorithm. The running time of preprocessing is $O(N^2)$ and of the main algorithm is $O(N)$. Therefore, the total running time of isometric reconciliation with our two algorithms and the two-pass algorithm \cite{chladek_thesis} is $O(N^2)$ for a rooted gene tree and $O(N^4 + P)$ for an unrooted gene tree.

Created algorithms were implemented into a source code by Chládek \cite{chladek_thesis} which resulted in an isometric reconciliation software with a command-line interface. We made several changes in the original source code and additions that we highlighted and described.

Eventually, we tested the implemented software on the simulated and real dataset and compared it to other software. The simulated dataset consisted of two clades with 1000 simulated gene families, where the gene trees are rooted. We run the software without and with rerooting the gene trees. Without rerooting the gene trees, we were testing the impact of different tolerance settings on the reconciliation. We found that the best tolerance needs to be by one decimal place shorter than the number of decimal places in the edge length in a rooted gene tree. Our software was the fastest in inferring the isometric reconciliation. With rerooting the gene tree, we were testing the impact of the tolerance and step settings on the reconciliation. We found the optimal tolerance and step settings for the simulated dataset. In both clades, our software infers the best solution compared to the other software two times. The real dataset consisted of 5351 alignments that we transform into unrooted trees. Our software had comparable good results with other software.




