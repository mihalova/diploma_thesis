\chapter*{List of appendices}
\addcontentsline{toc}{chapter}{List of Appendices}

Appendix A - Source code \\
Appendix B - User manual

\chapter*{Appendix B - User manual}

Our isometric reconciliation software has a command line interface.

\section*{Input}

The needed input for our software is a rooted species tree with exact branch lengths in Newick format and a rooted or unrooted gene tree with exact or inexact branch lengths in Newick format with mapping of gene tree leaves to the species tree leaves.

Inexact branch lengths are required in format, where minimal branch length and maximal branch length are separated with a dash \emph{"-"} such as \emph{"geneName:minimalLength-maximalLength"} shown in an example \emph{"cat:1.3785-2.3375"}.

We recognise two types of leaves mapping. The first type of leaves mapping is directly in the gene tree file, where each gene leaf name consists of the name of the gene and the name of the species to which it is mapping separated with underscore \emph{"\_"} such as \emph{"geneName\_speciesName"} shown in an example \emph{"cat12\_cat"}. The second type of leaves mapping is stored in an \emph{".smap"} file, where each gene name has assigned a species. It is a pattern-matching mapping and we consider only matching at prefix. For example, if the gene name starts always with \emph{"CAT"}, the mapping is \emph{"CAT*	cat"}, where the \emph{"*"} is optional postfix after the matching pattern and the gene leaf in a gene tree is \emph{"CAT12:0.8324"}.

\section*{Output}

The output of our software is a relations file or reconciled gene trees in Newick format.

The relations file recognise 4 types of relations for each node in gene tree:
\begin{itemize}
  \item \emph{"gene"} - for a gene node in a leaf,
  \item \emph{"spec"} - if the evolutionary event occurred on the gene node is speciation,
  \item \emph{"dup"} - if the evolutionary event occurred on the gene node is duplication,
  \item \emph{"loss"} - if the evolutionary event occurred on the gene node is gene loss.
\end{itemize}
Each relation type is written in a new line with information about the node separated with spaces, such as:
\begin{itemize}
  \item \emph{"gene geneName"}, where \emph{geneName} is for the name of the gene
  \item \emph{"spec leftChild rightChild"}, where \emph{leftChild} is the name of a left child of the speciation node and \emph{rightChild} is the name of a right child of the speciation node,
  \item \emph{"dup leftChild rightChild"}, where \emph{leftChild} is the name of a left child of the duplication node and \emph{rightChild} is the name of a right child of the duplication node,
  \item \emph{"loss node"}, where \emph{node} is the name of a node above which the loss occurred.
\end{itemize}
Each name of an internal node of a gene tree consists of genes presented in a subtree of that node separated by commas \emph{","}. If the the found solutions have the same relations, we print only the first and the number of reconciled trees with the same relations.

\section*{Settings}


\section*{Running the reconciliation}



