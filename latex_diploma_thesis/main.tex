\documentclass[12pt,a4paper,oneside]{book}
\usepackage[utf8]{inputenc}
\usepackage[english]{babel}
\usepackage[T1]{fontenc}
\usepackage[hidelinks]{hyperref}
\usepackage{amsmath}
\usepackage{amsfonts}
\usepackage{amssymb}
\usepackage{algorithm}
\usepackage[noend]{algpseudocode}
\usepackage{listings}
\usepackage{indentfirst}
\usepackage{titlesec}
\usepackage{tocloft}
\usepackage{setspace}
\usepackage{graphicx}
\graphicspath{ {./figures/} }
%\usepackage{natbib}
\usepackage[left=3.5cm,right=2cm,top=2.5cm,bottom=2.5cm]{geometry}
\setstretch{1.5} %riadkovanie 1.5
\titleformat{\chapter}{\normalfont\huge\bfseries}{\thechapter}{20pt}{\normalfont\huge\bfseries}
\setcounter{tocdepth}{3}
\setcounter{secnumdepth}{3}
\newtheorem{definition}{Definition}

\newcommand\mytitle{Software for isometric gene tree reconciliation}
\newcommand\mythesistype{Master's thesis}
\newcommand\myauthor{Bc. Dominika Mihálová}
\newcommand\myadvisor{doc. Mgr. Bronislava Brejová, PhD.}
\newcommand\myplacedate{Bratislava, 2020}
\newcommand\myuniversity{COMENIUS UNIVERSITY IN BRATISLAVA}
\newcommand\myfaculty{FACULTY OF MATHEMATICS, PHYSICS AND INFORMATICS}
\newcommand{\sub}[1]{$_{\text{#1}}$}
\newcommand{\reference}[1]{č.~\ref{#1}}
\newcommand{\imageHeight}{150px}
\renewcommand{\cftchapleader}{\cftdotfill{\cftdotsep}}

\begin{document}
%obal
\frontmatter
\thispagestyle{empty}
\noindent
\begin{minipage}{\textwidth}
	\begin{center}
		\textbf{\myuniversity \\
		\myfaculty}
	\end{center}
\end{minipage}

\vfill
\begin{center}
	\begin{minipage}{0.8\textwidth}
		\centering{\textbf{\Large\MakeUppercase{\mytitle}}} \\
		\centering{\mythesistype}
	\end{minipage}
\end{center}
\vfill
2020 \hfill
\myauthor
\eject 

%titulná strana
\thispagestyle{empty}
\noindent
\begin{minipage}{\textwidth}
	\begin{center}
		\textbf{\myuniversity \\
		\myfaculty}
	\end{center}
\end{minipage}
\vfill

\begin{center}
	\begin{minipage}{0.8\textwidth}
		\centering{\textbf{\Large\MakeUppercase{\mytitle}}} \\
		\centering{\mythesistype}
	\end{minipage}
\end{center}
\vfill
\begin{tabular}{l l}
	Study programme: & Applied Computer Science\\
	Field of study: & Computer Science\\
	Department: & Department of Computer Science\\
	Supervisor: & \myadvisor
\end{tabular}
\vfill
\noindent
\myplacedate \hfill
\myauthor
\eject 

%zadanie - pojde obrazok
\chapter*{Thesis assignment}
\vfill\eject 

%sk abstrakt
\chapter*{Abstrakt}

Cieľom diplomovej práce bolo implementovať softvér pre izometrickú rekonciliáciu génových stromov a experimentálne otestovať jeho presnosť na simulovaných aj reálnych biologických dátach. Práca je rozdelená do piatich kapitol.

Prvá kapitola je venovaná základnej terminológií z oblasti bioinformatiky a prehľadu rôznych prístupov k riešeniu problému rekonciliácie s názornými riešeniami tejto problematiky v podobe softvérov.

V druhej kapitole sa uvádza metodika práce a metódy skúmania použité na dosiahnutie cieľa.

Ďalšia časť charakterizuje implementovaný softvér. Predkladá sa návrh na softvér, špecifikujú sa jeho vlastnosti, opisujú sa implementované algoritmy, spôsob spracovania vstupov a následné výstupy.

Záverečná kapitola sa zaoberá opisom testovacej sady a následným testovaním implementovaného softvéru a jeho výsledkami.
\\\\
\textbf{Kľúčové slová:} izometrická rekonciliácia génového stromu, nepresné dĺžky hrán, fylogenetický strom
\vfill\eject 

%en abstrakt
\chapter*{Abstract}

The main goal of the diploma thesis was to implement software for isometric gene tree reconciliation and to experimentally evaluate its accuracy on simulated and real biological data. The thesis is divided into five chapters.

The first chapter presents the basic terminology in the field of bioinformatics and an overview of different approaches to solving the problem of reconciliation with concrete solutions to this problem in the form of software.

The second chapter shows the methodology of work and research methods used to achieve the main goal.

The next part characterizes the implemented software. A proposal for the software is presented, its properties are specified, the implemented algorithms are described, the method of input processing and subsequent outputs is described.

The final chapter deals with the description of the test set and subsequent testing of the implemented software and its results.
\\\\
\textbf{Keywords:} isometric gene tree  reconciliation, inexact branch lengths, phylogenetic tree
\vfill\eject  

%zoznam obrazkov
\listoffigures
\newpage

%obsah
\tableofcontents
\newpage

\mainmatter

%úvod
\chapter*{Introduction}
\addcontentsline{toc}{chapter}{Introduction}
\vfill\eject

\input chapter_1_overview.tex
\input chapter_2_methodology.tex
\input chapter_3_specification.tex
\input chapter_4_implementation.tex
\input chapter_5_results.tex

\backmatter

%záver
\chapter*{Conclusion}
\addcontentsline{toc}{chapter}{Conclusion}
\vfill\eject 

%príloha
\chapter*{Appendix}
\addcontentsline{toc}{chapter}{Appendix}
\vfill\eject 



\nocite{*}
\bibliographystyle{plain}
\bibliography{bibliography}

\end{document}